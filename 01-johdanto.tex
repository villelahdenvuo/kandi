\section{Johdanto}

JavaScript on ohjelmointikieli, joka suunniteltiin ensisijaisesti verkkosivujen tekijöille. Sen tavoite oli täydentää Java-ohjelmointikieltä ja HTML-merkkauskieltä. JavaScript mahdollistaa monimutkaisten Internet-selaimissa suoritettavien sovellusten kirjoittamisen ilman selainlaajennuksia~\cite{paolini1994netscape}.

Selainten suosio sovellusalustana on kasvanut ja samalla JavaScriptin käyttö on lisääntynyt paljon. Kasvua siivittää se, että kaikissa kuluttajatietokoneissa on jokin selain ja sovellusten käyttämiseen riittää verkkosivuilla vieraileminen. Käyttäjän ei tarvitse asentaa sovellusta ennen käyttämistä.

Selaimet ovat kehittyneet ja niiden käyttämiä teknologioita on standardisoitu. Näistä niin sanotuista Web-teknologioista, joihin JavaScript lasketaan, on tullut varteenotettavia vaihtoehtoja moderniin sovelluskehitykseen. Web-teknologioiden käyttö ei kuitenkaan rajoitu vain selaimiin. Niillä on toteutettu esimerkiksi palvelinsovelluksia sekä kokonaan ilman selainta toimivia sovelluksia, kuten Atom-tekstieditori~\cite{atom}.

JavaScript on dynaaminen oliopohjainen kieli, mutta se tukee myös imperatiivista ja funktionaalista ohjelmointityyliä. JavaScript tarjoaa siis monia tapoja toteuttaa sama asia~\cite[4.2.1.]{es6}. Muuttujat JavaScriptissä ovat dynaamisesti tyypitettyjä ja tämä helpottaa ohjelmakoodin kirjoittamista. Dynaamisuudesta seuraa kuitenkin myös ongelmia, sillä virtuaalikoneiden on vaikea ennustaa dynaamisia muutoksia ja tehdä järkeviä optimointeja, joilla suorituskykyä voidaan parantaa~\cite{Ahn2014}.

Modernit JavaScript-virtuaalikoneet ovat kehittyneet paljon viime vuosina. Niihin on toteutettu monimutkaisia ja kekseliäitä menetelmiä suorituskyvyn parantamiseksi. Suuri osa virtuaalikoneiden optimoinneista perustuu oletukseen, että dynaamisuudesta huolimatta ohjelmat käyttäytyvät yleensä melko staattisesti, kun niitä suoritetaan. Keräämällä tyyppitietoa suorituksen aikana, virtuaalikoneet pystyvät esimerkiksi luomaan optimoitua konekoodia osalle lähdekoodista. Tämä optimoitavuus edellyttää, että ohjelmoija tietää miten asiat kannattaa toteuttaa. \textbf{MIKSI?}

%Esimerkkejä tälläisistä ovat muun muassa \textit{piiloluokat} (hidden classes)~\cite{v8design} ja erilaiset suorituksenaikaiset optimoivat kääntäjät. Piiloluokat ovat staattisia luokkia, joita virtuaalikone käyttää objektien kuvaamiseen. Jos objektia muutetaan, esimerkiksi lisäämällä uusi kenttä, luodaan sille uusi piiloluokka. Piiloluokkien hyödyt tulevat esiin, jos ohjelmassa on paljon samanlaisia objekteja, jotka voivat käyttää samaa piiloluokkaa.

Kieltä kuitenkin kehitetään jatkuvasti ja siihen on tuotu muun muassa luokka- ja moduulijärjestelmät~\cite[14.5.~ja~15.2.]{es6}, jotka pyrkivät yhtenäistämään erilaisia toteutustapoja ja mahdollistavat tällä tavoin aikaisempaa paremman ennustettavuuden ja suorituskyvyn.

JavaScriptin standardisoiminen, sen käytön lisääntyminen ja selainvalmistajien keskinäinen kilpailu suorituskyvystä on parantanut kielen asemaa ja mainetta. JavaScriptin rooli on muuttunut skriptikielestä yleiskäyttöiseksi ohjelmointikieleksi~\cite[4.]{es6}. Kielen tulevaisuus näyttää lupaavalta. Siihen on tulossa paljon ohjelmointia helpottavia ominaisuuksia ja tapoja välttää yleisiä sudenkuoppia, joihin varsinkin aloittelevat ohjelmoijat usein törmäävät.