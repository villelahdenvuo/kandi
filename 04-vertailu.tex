\section{Toteutusten vertailua}

Virtuaalikonetoteutukset ovat ottaneet käyttöön toistensa optimointimenetelmiä, mutta niiden kehittäjien erilaiset ajattelutavat ja perinteet ovat ohjanneet virtuaalikoneiden arkkitehtuureja ja käytäntöjä. Toteutuksista löytyy siis yhtäläisyyksiä sekä eroavaisuuksia, jotka vaikuttavat niiden suorituskykyyn, vaikka JavaScriptin toiminta onkin tarkasti määritelty. Vertailuun on valittu Googlen V8-, Applen JavaScriptCore-, Mozillan SpiderMonkey- ja Microsoftin Chakra-virtuaalikoneet, koska niistä löytyy parhaiten tietoa paitsi Chakrasta. Se on toistaiseksi suljettua lähdekoodia, mutta silti mielenkiintoinen ja merkittävä käyttäjämäärällisesti. Microsoft on avaamassa sen lähdekoodin vuoden 2016 alussa~\cite{chakraopen}.

\subsection{Suoritusarkkitehtuuri}

Ensimmäinen kiinnostava tieto on, mitä virtuaalikone tekee korkealla tasolla koodille suorittaakseen sitä. Virtuaalikoneet eivät ole hylänneet tulkkejaan, vaikka niihin on lisätty JIT-kääntäjiä. Esimerkiksi JavaScriptCoren tulkki on kirjoitettu kokonaan uudelleen paljon edeltäjäänsä paremmaksi ja sille on annettu uusi nimi LLInt~\cite{llint}. Ironista kyllä vaikka Google ei alunperin toteuttanut lainkaan tulkkia V8-virtuaalikoneeseensa, se on nyt kehittämässä tulkkia nimeltä Ignition~\cite{v8ignition}. Uusi tulkki perustuu JavaScriptCoren LLInt-tulkkiin.

Perustelut muutokselle ovat ymmärrettäviä, sillä uutta tulkkia aiotaan käyttää ainakin alkuun ensisijaisesti mobiililaitteissa, joissa on rajoitettu määrä muistia ja heikompi suorituskyky. Tällaisilla laitteilla kääntäminen konekoodiksi ennen suorittamista on yksinkertaisesti liian hidasta, minkä takia halutaan nopeasti käynnistyvä tulkki~\cite{v8ignition}. Lisäksi konekoodiksi käännetty koodi vie enemmän muistia kuin tavukoodimuotoinen koodi.

Aikaisemmin V8 on pitänyt JavaScript-lähdekoodia parhaana esitysmuotona ohjelmalle ja siitä on jäsennetty abstrakti syntaksipuu joka kerta ennen kääntämistä. Uusi tavukoodimuoto voisi kuitenkin toimia myös parempana välikielenä kääntäjille, jolloin alkuperäistä lähdekoodia ei tarvitsisi jäsentää uudelleen joka kerta.

Myös Chakra-virtuaalikone käyttää tulkkia suorituksen alussa. Uusimmassa Chakran versiossa on optimoivan JIT-kääntäjän lisäksi yksinkertainen JIT-kääntäjä, joka ei tee monimutkaisia optimointeja mahdollistaen nopeamman käännöksen konekoodiksi~\cite{chakra}. Saatavilla olevien tietojen mukaan Chakra ei tue aktivaatiotietueiden manipulointia, joten optimoitu koodi otetaan käyttöön vasta kun funktiota kutsutaan uudestaan. Chakran kehittäjät ovat pyrkineet hyödyntämään laitteistoa mahdollisimman paljon rinnakkaistamalla käännös- ja roskienkeruuoperaatioita. Myös SpiderMonkey ja V8 suorittaa käännöstyötä rinnakkaisesti ohjelman suorituksen kanssa.

SpiderMonkeyn kehittäjät ovat olleet myös ahkeria ja virtuaalikone on nähnyt jo viisi erilaista JIT-kääntäjää: TraceMonkey, JägerMonkey, IonMonkey, OdinMonkey ja Baseline Compiler (alkutilannekääntäjä)~\cite{monkeys}. Kaikki niistä ei ole enää käytössä, vaan osa on korvannut aikaisempia ja osa on siirtynyt eri vaiheeseen suorituksessa. Esimerkiksi alkutilannekääntäjä on korvannut JägerMonkeyn, joka korvasi sitä edeltävän TraceMonkeyn.

SpiderMonkey käyttää yhä tulkkia suorituksen alussa, mutta pyrkii kääntämään koodin mahdollisimman nopeasti alkutilannekääntäjällään~\cite{baseline}, IonMonkey keskittyy erityisesti paljon kutsuttujen funktioiden kovakouraiseen optimointiin ja OdinMonkey-kääntäjää käytetään vain asm.js-muotoisen koodin kääntämiseen etukäteen. Asm.js:stä kerrotaan lisää Tulevaisuus-luvussa.

SpiderMonkey ei ole ainoa virtuaalikone, joka on nähnyt useita virtuaalikoneita. V8:n kehittäjät rakentavat uutta JIT-kääntäjää, jota he kutsuvat nimellä TurboFan~\cite{turbofan}. Sen on tarkoitus korvata nykyinen optimoiva kääntäjä, Crankshaft, ja olla sitä helpommin jatkokehitettävä. Myös JavaScriptCoren kehittäjät valmistelevat FTL JIT -nimistä (Fourth Tier LLVM JIT) kääntäjää neljänneksi kääntäjäksi~\cite{ftljit}. Nimensäkin perusteella se hyödyntää LLVM kääntäjäinfrastruktuuria, jonka tarkoitus on helpottaa kääntäjien tekemistä tarjoamalla valmis rajapinta. Kääntäjän riittää tuottaa LLVM-yhteensopivaa välikieltä ja LLVM hoitaa sen optimoinnin ja kääntämisen konekoodiksi.

\subsection{Tyyppien päättely}

Konekoodista ei yksinkertaisesti saa nopeaa, jos jokaisen operaation kohdalla pitää tarkistaa muuttujien tyypit, suorittaa erilaiset tyyppimuunnokset ja varmistaa kaikenlaiset poikkeustapaukset. Tämän takia kaikki vertailun virtuaalikoneet käyttävät jonkinlaista tyyppijärjestelmää, kuten piiloluokkia, taustalla ja päättelevät muuttujien tyyppejä suoritusaikana kerätyn tiedon perusteella.

V8-virtuaalikoneen tapauksessa kerätty tyyppitieto tallennetaan sisällytettyihin välimuisteihin, jotka ovat osa generoitua konekoodia (eli suoritettavaa muistia)~\cite{llint}. JavaScriptCoren LLInt-tulkki kerää myös tyyppitietoa sisällytetyillä välimuisteilla, mutta se ei tallenna tyyppitietoa suoritettavan muistin sekaan.

JIT-kääntäjät pystyvät sitten hyödyntämään tätä tyyppitietoa luodessaan optimoituja versioita funktioista. Mutta kuten sisällytetyn välimuistin luvussa mainittiin, täytyy silti varautua poikkeuksiin ja mahdollisesti vaihtaa optimoitu koodi takaisin epäoptimoituun versioon.

Chakran kehittäjät ovat vieneet optimoinnin askelta pidemmälle. Chakra samastaa tyyppejä niin sanotulla ''equivalent object type specilization''-menetelmällä~\cite{chakra}. Menetelmän avulla esimerkiksi aikaisemmin mainitut Piste- ja Ympyräoliot pystyvät käyttämään samaa sisällytettyä välimuistia funktion \texttt{getX} hakupaikassa, vaikka niillä on eri piiloluokat. Se onnistuu, koska niiden rakenne on tarpeeksi lähellä toisiaan. Esimerkin tapauksessa ne ovat ekvivalentit tyypeiltään, koska molemmilla on ominaisuus \texttt{x} samassa siirtymässä.

\begin{comment}
\begin{itemize}
\item \url{https://wiki.mozilla.org/TypeInference}
\item SpiderMonkey: Staattisen ja dynaamisen päättelyn yhdistelmä!
\item \url{https://trac.webkit.org/wiki/JavaScriptCore#TypeInference}
\end{itemize}
\end{comment}