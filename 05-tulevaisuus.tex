\section{Tulevaisuus}

Brendan Eichin mukaan JavaScriptistä on tullut jo kliseen omaisesti ''Webin konekieli''~\cite{webassembly}. JavaScript-ohjelmia suoritetaan käytännössä jokaisella alustalla ja kehittäjät ovat alkaneet tehdä kääntäjiä, jotka kääntävät muita ohjelmointikieliä, vanhoja tai uusia, JavaScriptiksi. Tämä lisää painetta parantaa virtuaalikoneiden suorituskykyä ja lisätä matalamman tason rajapintoja kääntäjäohjelmoijien hyödynnettäväksi.

Asm.js~\cite{asmjs} on epävirallinen standardi JavaScriptin osajoukosta, jota on mahdollista kääntää tehokkaaksi konekoodiksi. Sen idea on olla toimivaa JavaScript-koodia, mutta mahdollistaa tehokas kääntäminen suoraan konekoodiksi. Se onnistuu kertomalla virtuaalikoneelle, että koodi on asm.js-muotoista ja tarjoamalla tyyppivihjeitä. Tässä esimerkki kokonaislukujen summafunktiosta asm.js-koodina:
\begin{lstlisting}
"use asm"; // Kerrotaan, että koodi on asm.js-muotoista.
function kokonaislukujenSumma(x, y) {
  x = x|0; y = y|0;
  return (x + y)|0;
}
\end{lstlisting}
Esimerkissä kokonaislukuparametrista annetaan vihje tekemällä funktion alussa bittitason tai-operaatio: \texttt{x = x|0}. Jos \texttt{x}:n arvo on \texttt{undefined} tai jokin olio, operaatio muuttaa sen kielen standardin mukaisesti kokonaisluvuksi. Jos virtuaalikone tukee asm.js-kääntämistä, tyyppivihje kertoo virtuaalikoneelle, että parametri on aina kokonaisluku. Virheelliset tyypit voidaan huomata staattisilla työkaluilla suorittamatta koodia. JavaScriptin aritmetiikka toimii aina liukuluvuilla, joten sen takia myös summan tulos pitää muuttaa kokonaisluvuksi.

Mozillan OdinMonkey-kääntäjän lisäksi myös Googlen V8-virtuaalikoneen kehittäjät suunnittelevat tukea asm.js-optimoinneille TurboFan-kääntäjän avulla~\cite{turbofan}. Asm.js:n tueksi JavaScriptiin on tuotu lisää suorituskykyä parantavia toimintoja, kuten \textit{SIMD-käskyt}~\cite{webassembly}. SIMD tulee sanoista \textit{Single Instruction Multiple Data}, joka tarkoittaa suomeksi: ''Yksi käsky, useita data-alkioita''. SIMD-käskyjen avulla pystytään hyödyntämään prosessorien mahdollisuutta käsitellä useita data-alkioita, kuten vektoreita, yhdellä konekäskyllä.

Yleensä JavaScript-ohjelmat käyttävät vain yhtä säiettä, joten ne eivät hyödynnä moniytimisiä prosessoreja kovin hyvin. Vaikka kielessä on jo Web Worker -rajapinta, joka mahdollistaa ohjelman jakamisen rinnakkaisiin tehtäviin. Tehtävien välinen kommunikaatio tapahtuu viestinvälityksellä, joka on melko hidasta. Tukea rinnakkaisohjelmoinnille halutaan parantaa tuomalla \textit{SharedArrayBuffer}-rajapinta, eli jaettu taulukkopuskuri, ja atomiset operaatiot. Nämä yhdessä mahdollistavat matalan tason rinnakkaisohjelmoinnin, joka hyödyttää varsinkin raskasta laskentaa vaativia sovelluksia kuten pelejä.

Asm.js alkoi kokeellisena toteutuksena, mutta nyt selainvalmistajat ja standardoijat kehittävät yhdessä virallista Webin konekieltä, jota he kutsuvat nimellä WebAssembly~\cite{webassembly}. Sen on tarkoitus tarjota matalan tason binääriformaatti, jota kääntäjät voivat tuottaa. Koska WebAssembly on tiiviissä binäärimuodossa, sitä ei tarvitse purkaa, kuten pakattua JavaScript-koodia, eikä jäsentää uudelleen selaimessa. WebAssemblyn tavoite ei ole korvata JavaScript-koodia ja nykyistä kehitystapaa, vaan tarjota parempi tuki myös käännetyille ohjelmille, jotka aikaisemmin ovat toimineet helposti haavoittuvina selainlaajennuksina.

% TODO: SoundScript in V8, Transpilers!