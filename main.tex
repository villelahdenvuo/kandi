% --- Template for thesis / report with tktltiki2 class ---
% 
% last updated 2013/02/15 for tkltiki2 v1.02

\documentclass[finnish]{tktltiki2}

% tktltiki2 automatically loads babel, so you can simply
% give the language parameter (e.g. finnish, swedish, english, british) as
% a parameter for the class: \documentclass[finnish]{tktltiki2}.
% The information on title and abstract is generated automatically depending on
% the language, see below if you need to change any of these manually.
% 
% Class options:
% - grading                 -- Print labels for grading information on the front page.
% - disablelastpagecounter  -- Disables the automatic generation of page number information
%                              in the abstract. See also \numberofpagesinformation{} command below.
%
% The class also respects the following options of article class:
%   10pt, 11pt, 12pt, final, draft, oneside, twoside,
%   openright, openany, onecolumn, twocolumn, leqno, fleqn
%
% The default font size is 11pt. The paper size used is A4, other sizes are not supported.
%
% rubber: module pdftex

% --- General packages ---

\usepackage[utf8]{inputenc}
\usepackage[T1]{fontenc}
\usepackage{lmodern}
\usepackage{microtype}
\usepackage{amsfonts,amsmath,amssymb,amsthm,booktabs,color,enumitem,graphicx}
\usepackage[pdftex,hidelinks]{hyperref}

% Automatically set the PDF metadata fields
\makeatletter
\AtBeginDocument{\hypersetup{pdftitle = {\@title}, pdfauthor = {\@author}}}
\makeatother

% babelbib for non-english bibliography using bibtex
\usepackage[fixlanguage]{babelbib}
\selectbiblanguage{finnish}

% Babelbib doesn't support finnish ordinals for example edition = 6 -> kuudes painos.
\declarebtxcommands{finnish}{
  \def\btxnumeralshort#1{
    #1.}
  \def\btxnumerallong#1{
    \ifnumber{#1}{
      \ifcase#1 nollas\or ensimmäinen\or toinen\or kolmas\or neljäs\or viides\or
        kuudes\or seitsemäs\or kahdeksas\or yhdeksäs\or kymmenes\else
        #1.
      \fi
    }{#1}}
}

% Remove [brackets] around keys.
\makeatletter
\renewcommand\@biblabel[1]{\hfill #1.}
\makeatother

% Try not to break small words. (että, jotta, etc.) (range: 0 - 10 000)
\pretolerance=1000

% add bibliography to the table of contents
\usepackage[nottoc]{tocbibind}
% tocbibind renames the bibliography, use the following to change it back
\settocbibname{Lähteet}

% --- Theorem environment definitions ---

\newtheorem{lau}{Lause}
\newtheorem{lem}[lau]{Lemma}
\newtheorem{kor}[lau]{Korollaari}

\theoremstyle{definition}
\newtheorem{maar}[lau]{Määritelmä}
\newtheorem{ong}{Ongelma}
\newtheorem{alg}[lau]{Algoritmi}
\newtheorem{esim}[lau]{Esimerkki}

\theoremstyle{remark}
\newtheorem*{huom}{Huomautus}

% --- tktltiki2 options ---

\title{JavaScript ja virtuaalikoneet}
\author{Ville Lahdenvuo}
\date{\today}
\level{Aine}
%\level{Kandidaatintutkielma}
\abstract{Tiivistelmä.}

% The following can be used to specify keywords and classification of the paper:

\keywords{JavaScript, virtuaalikone, suorituskyky}

% classification according to ACM Computing Classification System (http://www.acm.org/about/class/)
\classification{
  \textbf{Software and its engineering $\rightarrow$ Virtual machines} \\
  Software and its engineering $\rightarrow$ Very high level languages
}

% If the automatic page number counting is not working as desired in your case,
% uncomment the following to manually set the number of pages displayed in the abstract page:
%
% \numberofpagesinformation{16 sivua + 10 sivua liitteissä}

\begin{document}

% --- Front matter ---

\frontmatter      % roman page numbering for front matter

\maketitle        % title page
\makeabstract     % abstract page

\tableofcontents  % table of contents

% --- Main matter ---

\mainmatter       % clear page, start arabic page numbering

% Set bigger space between lines.
\renewcommand{\baselinestretch}{1.5}
\selectfont

%\noindent
JavaScript on ohjelmointikieli, joka suunniteltiin ensisijaisesti verkkosivujen tekijöille, joilla ei välttämättä ole paljon ohjelmointikokemusta. Sen idea oli täydentää Java-ohjelmointikieltä ja HTML-merkkauskieltä. Se mahdollistaa monimutkaisten Internet-selaimissa suoritettavien sovellusten kirjoittamisen~\cite{paolini1994netscape}.

Selainten suosio sovellusalustana on kasvanut ja samalla JavaScriptin käyttö on lisääntynyt räjähdysmäisesti. Se ei ole yllättävää, sillä kaikissa kuluttajatietokoneissa on jokin selain. Lisäksi sovellusten jakaminen on helppoa, koska siihen riittää pelkkä URL-osoite.

Selaimet ovat kehittyneet ja niiden käyttämiä teknologioita on standardoitu. Tämä on auttanut tekemään JavaScriptistä varteenotettavan vaihtoehdon moderniin sovelluskehitykseen. JavaScriptin käyttö ei kuitenkaan rajoitu vain selaimiin, vaan sitä käytetään myös palvelinsovelluksissa sekä muissa ilman selainta toimivissa sovelluksissa, kuten esimerkiksi Atom-tekstieditorissa~\cite{atom}.

JavaScript on dynaaminen oliopohjainen kieli, mutta se tukee myös imperatiivista ja funktionaalista ohjelmointityyliä. JavaScript siis tarjoaa monta tapaa toteuttaa samoja asioita~\cite[Osio 4.2.1.]{es6}. Muuttujat JavaScriptissä ovat tyypittömiä ja tämä helpottaa ohjelmointia tekemällä muuttujien käytöstä vapaampaa. Usein tästä dynaamisuudesta ja tyypittömyydestä seuraa kuitenkin myös ongelmia, sillä virtuaalikoneiden on vaikea ennustaa dynaamisia muutoksia ja tehdä järkeviä optimointeja, joilla suorituskykyä saisi parannettua~\cite{Ahn2014}.

Kieltä kuitenkin kehitetään jatkuvasti ja siihen ollaan tuomassa muun muassa luokka- ja moduulijärjestelmät~\cite[Osiot~14.5.~ja~15.2.]{es6}, jotka toivottavasti yhtenäistävät toteutustapoja ja mahdollistavat tällä tavoin paremman ennustettavuuden ja suorituskyvyn.

Modernit JavaScript-virtuaalikoneet ovat kehittyneet paljon viime vuosina ja niihin on toteutettu monimutkaisia ja kekseliäitä menetelmiä suorituskyvyn parantamiseksi. Esimerkkejä tälläisistä ovat muun muassa piiloluokat (hidden classes) ja erilaiset ajonaikaiset optimoivat kääntäjät.

Iso osa virtuaalikoneiden parannuksista perustuvat oletukseen, että vaikka kieli itsessään on dynaaminen, ohjelmat kuitenkin käyttäytyvät yleensä melko staattisesti. Keräämällä tyyppitietoa suorituksen aikana, virtuaalikoneet pystyvät esimerkiksi luomaan optimoitua konekoodia osalle lähdekoodista.

On selvää, että JavaScriptin standardoiminen, sen käytön lisääntyminen ja selainvalmistajien keskinäinen kilpailu suorituskyvystä on parantanut kielen asemaa ja mainetta. Sanotaan, että JavaScriptin käyttö on siirtynyt skriptikielestä yleiskäyttöiseksi ohjelmointikieleksi~\cite[Osio~4.]{es6}. Kieltä pääsee myös kokeilemaan helposti, koska useimpien selainten mukana tulee kokonainen kehitysympäristö.
\section{Johdanto}

JavaScript on ohjelmointikieli, joka suunniteltiin ensisijaisesti verkkosivujen tekijöille. Sen tavoite oli täydentää Java-ohjelmointikieltä ja HTML-merkkauskieltä. JavaScript mahdollistaa monimutkaisten Internet-selaimissa suoritettavien sovellusten kirjoittamisen ilman selainlaajennuksia~\cite{paolini1994netscape}.

Selainten suosio sovellusalustana on kasvanut ja samalla JavaScriptin käyttö on lisääntynyt paljon. Kasvua siivittää se, että kaikissa kuluttajatietokoneissa on jokin selain ja sovellusten käyttämiseen riittää verkkosivuilla vieraileminen. Käyttäjän ei tarvitse asentaa mitään ennen sovelluksen käyttöä.

Selaimet ovat kehittyneet ja niiden käyttämiä teknologioita on standardisoitu. Näistä niin sanotuista Web-teknologioista, joihin JavaScript lasketaan, on tullut varteenotettava vaihtoehto moderniin sovelluskehitykseen. Web-teknologioiden käyttö ei kuitenkaan rajoitu vain selaimiin. Niillä on toteutettu esimerkiksi palvelinsovelluksia sekä kokonaan ilman selainta toimivia sovelluksia, kuten Atom-tekstieditori~\cite{atom}.

JavaScript on dynaaminen oliopohjainen kieli, mutta se tukee myös imperatiivista ja funktionaalista ohjelmointityyliä. JavaScript tarjoaa siis monia tapoja toteuttaa sama asia~\cite[4.2.1.]{es6}. Muuttujat JavaScriptissä ovat dynaamisesti tyypitettyjä ja tämä helpottaa ohjelmakoodin kirjoittamista. Dynaamisuudesta seuraa kuitenkin myös ongelmia, sillä virtuaalikoneiden on vaikea ennustaa dynaamisia muutoksia ja tehdä järkeviä optimointeja, joilla suorituskykyä voitaisiin parantaa~\cite[s.~497]{Ahn2014}.

Modernit JavaScript-virtuaalikoneet ovat kehittyneet paljon viime vuosina. Niihin on toteutettu monimutkaisia ja kekseliäitä menetelmiä suorituskyvyn parantamiseksi. Suuri osa virtuaalikoneiden optimoinneista perustuu oletukseen, että dynaamisuudesta huolimatta ohjelmat käyttäytyvät suorituksen aikana yleensä melko staattisesti. Keräämällä tyyppitietoa suorituksen aikana, virtuaalikoneet pystyvät esimerkiksi luomaan optimoitua konekoodia osalle lähdekoodista. Optimoitavuus edellyttää, että ohjelmoija tietää miten asiat kannattaa toteuttaa. Jos hyödyntää dynaamisuutta liikaa, voi helposti tehdä koodia, jota virtuaalikone ei pysty optimoimaan.

%Esimerkkejä tälläisistä ovat muun muassa \textit{piiloluokat} (hidden classes)~\cite{v8design} ja erilaiset suorituksenaikaiset optimoivat kääntäjät. Piiloluokat ovat staattisia luokkia, joita virtuaalikone käyttää objektien kuvaamiseen. Jos objektia muutetaan, esimerkiksi lisäämällä uusi kenttä, luodaan sille uusi piiloluokka. Piiloluokkien hyödyt tulevat esiin, jos ohjelmassa on paljon samanlaisia objekteja, jotka voivat käyttää samaa piiloluokkaa.

JavaScriptiä kuitenkin kehitetään jatkuvasti ja siihen on tuotu muun muassa luokka- ja moduulijärjestelmät~\cite[14.5.~ja~15.2.]{es6}. Nämä auttavat yhtenäistämään erilaisia toteutustapoja ja mahdollistavat tällä tavoin aikaisempaa paremmin ennustettavan käytöksen. Kun käytetään luokkasyntaksia, tulee käytettyä yhtenäistä tapaa muodostaa objekteja. Ennustettavuudesta seuraa parempi optimoitavuus ja suorituskyky~\cite[s.~497]{Ahn2014}.

JavaScriptin standardisoiminen, sen käytön lisääntyminen ja selainvalmistajien keskinäinen kilpailu suorituskyvystä on parantanut kielen asemaa ja mainetta. JavaScriptin rooli on muuttunut skriptikielestä yleiskäyttöiseksi ohjelmointikieleksi~\cite[4.]{es6}. Kielen tulevaisuus näyttää lupaavalta. Siihen on tullut paljon ohjelmointia helpottavia ominaisuuksia ja tapoja välttää yleisiä sudenkuoppia, joihin varsinkin aloittelevat ohjelmoijat usein törmäävät.
\section{Virtuaalikoneiden toiminta}

Virtuaalikone on ohjelma, joka tarjoaa todellisen tai hypoteettisen laitteen toiminnallisuuksia muille ohjelmille hyödyntäen sitä suorittavan \textit{isäntäjärjestelmän} abstraktioita. Virtuaalikone voi esimerkiksi virtualisoida optista asemaa käyttämällä isännän tiedostojärjestelmää hyväksi, jolloin virtuaalikoneessa suoritettava ohjelma luulee esimerkiksi lukevansa optista levyä, kun todellisuudessa tieto tulee kiintolevyltä.

\subsection{Kahdenlaisia virtuaalikoneita}

Virtuaalikoneita on kahdenlaisia, \textit{järjestelmä-} ja \textit{prosessivirtuaalikoneita}~\cite[s.~33]{vms}. Järjestelmävirtuaalikone tarjoaa kokonaisen käyttöjärjestelmän palvelut toisin kuin prosessivirtuaalikone, joka tarjoaa vaan yhden prosessin suorittamista varten tarvittavat palvelut. Tässä tutkielmassa virtuaalikoneella tarkoitetaan JavaScriptillä toteutettuja ohjelmia suorittavaa prosessivirtuaalikonetta.

Yksi suurimmista virtuaalikoneiden höydyistä on se, että ohjelma tarvitsee kääntää usean alustan sijaan yhdelle virtuaalikoneelle, jolloin se toimii kaikilla niillä alustoilla, joille kyseinen virtuaalikone on toteutettu. Virtuaalikoneessa suoritettava ohjelma pääsee käsiksi vain virtuaalikoneen tarjoamiin palveluihin, jolloin ohjelmat on helpompi eristää käyttöjärjestelmästä ja laitteistosta~\cite[s.~36]{vms}. Pahantahtoisen ohjelman on siis löydettävä haavoittuvuus sekä virtuaalikoneesta että sen isännästä.

\subsection{JavaScript-virtuaalikoneen anatomia}

Ensimmäisen JavaScript-virtuaalikoneen nimi on SpiderMonkey~\cite{spidermonkey}. Se toteutettiin Netscape-selainta varten vuonna 1995. Nykyään sitä ylläpitää Mozilla ja sitä käytetään muun muassa Mozillan Firefox-selaimessa. Nykyinen SpiderMonkey on kehittynyt paljon, mutta silti se koostuu kolmesta fundamentaalisesta komponentista: \textit{kääntäjä}, \textit{tulkki} ja \textit{roskienkerääjä}~\cite{spidermonkeydesign}.

Kääntäjä huolehtii koodin \textit{jäsentämisestä} (parsing) ja kääntämisestä \textit{tavukoodiksi}. Virtuaalikoneen tavukoodi on verrattavissa todellisen koneen konekoodiin. Sitä on helpompi käsitellä ohjelmallisesti kuin tekstimuotoista ohjelmakoodia, sillä se on yksinkertaisempaa syntaksiltaan, joskin runsassanaisempaa.

Tulkin tehtävä on suorittaa tavukoodia. Tulkki siis lukee tavukoodia ja kutsuu tarvittavia käskyjä isäntäjärjestelmässä. Tulkista on siis oltava oma versionsa jokaista erilaista alustaa varten. Yksi tulkin eduista on, että se on joustavampi kuin todellinen prosessori ja siihen voi toteuttaa monimutkaisempia tavukoodikäskyjä.

Roskienkerääjän tehtävä on yksinkertaisesti poistaa muistista muuttujat ja oliot, joihin ohjelmassa ei enää viitata. Roskienkerääjän ansiosta ohjelmoijan ei tarvitse vapauttaa muistia itse, vaan järjestelmä hoitaa muistinhallinnan automaattisesti. Automaattinen muistinhallinta vähentää virheiden määrää, kuten muistivuotoja, mutta ei poista kaikkia ongelmia~[lähde?].

\subsection{Suorituksenaikainen kääntäminen}

Valitettavasti tulkit ovat hitaita, tai ainakin hitaampia kuin haluttaisiin. Riippumatta toteutustavasta, tulkki joutuu aina tekemään useita konekäskyjä yhden tavukoodikäskyn suorittamiseksi~\cite[s.~35]{vms}

Vuonna 2008 Google julkaisi uuden selaimen, Google Chromen, jonka oli tarkoitus parantaa verkkosovellusten käyttökokemusta~\cite{chromepress}. Googlen kiinnostus käyttökokemuksen ja ennen kaikkea suorituskyvyn parantamisesta on ymmärrettävää, sillä yhtiöllä on paljon verkkopalveluita, jotka hyötyvät hyvästä suorituskyvystä. Näistä syistä Google päätyi toteuttamaan oman virtuaalikoneen V8:n.

Mielenkiintoisen V8:sta tekee se, että siinä ei ole lainkaan tulkkia. Sen sijaan V8 kääntää koodin nopeasti suoraan konekoodiksi ennen suorittamista. Nopean kääntämisen saavuttamiseksi, se ei tee monimutkaisia optimointeja vielä ensimmäisessä käännösvaiheessa.

V8:n innoittamana muut virtuaalikonetoteutukset ovat muuttaneet toimintaansa siten, että tulkkia käytetään vain suorituksen alkuvaiheessa ja koodi pyritään kääntämään konekoodiksi mahdollisimman nopeasti \textit{suorituksenaikaisella kääntäjällä} eli \textit{JIT-kääntäjällä} (Just-In-Time compiler)~[lähde?].

Varsinkin usein kutsutut, niin sanotut ``kuumat'' funktiot, pyritään kääntämään optimoiduksi konekoodiksi. Tätä varten käytetään erillistä optimoivaa JIT-kääntäjää, joka on hitaampi kuin ensimmäinen \textit{lähtötilannekääntäjä} (baseline compiler), mutta tuottaa suorituskykyisempää konekoodia. JavaScript-koodin optimointi osoittautuu kuitenkin hankalaksi.

\subsection{Optimoinnin ongelmat}

Dynaamisesti tyypitetty konekoodi vaatii paljon tyyppitarkastuksia ja poikkeustapauksia riippuen muuttujien tyypeistä. On kuitenkin huomattu, että ohjelmat käyttäytyvät melko ennustettavalla tavalla. Etenkin usein kutsuttuja funktiota kutsutaan usein samantyyppisillä parametreilla.

Virtuaalikoneiden ei kannata suoraan generoida optimoitua konekoodia JavaScript-ohjelmista, sillä niillä ei ole tietoa muuttujien tyypeistä. Prosessorin kannalta on hyvin tärkeää tietää tehdäänkö jokin operaatio kokonaisluvuille, liukuluvuille tai kenties merkkijonoille. Lisäksi ei ole järkevää käyttää paljon aikaa koodin optimointiin, jos se suoritetaan vain muutamia kertoja.

Virtuaalikoneet aloittavat keräämällä tietoa ohjelman käyttäytymisestä. Tiedon kerääminen hoidetaan usein tulkissa ja V8:n lähtötilannekääntäjä lisää generoituun konekoodiin käskyjä keräämään tietoa ohjelman käyttämistä tyypeistä~\cite{v8compilers}.

Avoimen lähdekoodin WebKit-projekti sisältää JavaScriptCore-nimisen virtuaalikoneen, jota käytetään esimerkiksi Applen Safari selaimessa. JavaScriptCore koostuu tulkista, yksinkertaisesta JIT-kääntäjästä sekä Googlen V8:n innoittamana optimoivasta JIT-kääntäjästä, jota he kutsuvat nimellä \textit{DFG-JIT}. DFG tulee sanoista \textit{Data Flow Graph}, joka kuvaa ohjelman suorituksenaikaisen tyyppitiedon tallentavaa tietorakennetta. Eli muiden virtuaalikoneiden tapaan, JavaScriptCore kerää ensin tyyppitietoa ja sitten generoi optimoitua konekoodia~\cite{javascriptcore}.

Automaattinen roskienkeräys on todella kätevä toiminnallisuus ohjelmoijan kannalta, mutta sen toteuttaminen hyvin on haastavaa. Virtuaalikoneen täytyy pysäyttää ohjelman suoritus ja käydä läpi muistin sisältö vapauttaen muistialueita, jotka eivät ole enää käytössä. Selaimen tapauksessa tämä voi aiheuttaa verkkosovelluksen hidastumista. Hidastuminen huonontaa käyttökokemusta, varsinkin jos kyseessä on interaktiivinen ohjelma tai animaatio.

%https://blog.chromium.org/2010/12/new-crankshaft-for-v8.html
%An optimizing compiler which recompiles and optimizes hot code identified by the runtime profiler. It uses static single assignment form to perform optimizations such as loop-invariant code motion, linear-scan register allocation and inlining. The optimization decisions are based on type information collected while running the code produced by the base compiler.
\section{Suorituskyky}

On selvää, että JavaScript-koodin kääntäminen konekoodiksi on aina nopeampaa kuin tulkkaaminen, sillä tulkkaamiseen kuluu aina enemmän konekäskyjä kuin valmiin koodin suorittamiseen. Toki kääntäminen tuo ongelmaksi hitaan käännösvaiheen. Tämän takia modernit virtuaalikoneet sisältävät useamman kuin yhden kääntäjän eritasoisilla optimoinneilla. Tässä luvussa esitellään muutamia keinoja, joilla JavaScript-virtuaalikoneet ovat parantaneet suorituskykyä.

\subsection{Piiloluokat}

\textit{Piiloluokat} (hidden classes)~\cite{v8design} ovat staattisia luokkia, joita virtuaalikone käyttää JavaScript-objektien kuvaamiseen. Jos objektia muutetaan, esimerkiksi lisäämällä uusi kenttä, luodaan sille uusi piiloluokka. Piiloluokkien hyödyt tulevat esiin, jos ohjelmassa on paljon samanlaisia objekteja, jotka voivat hyödyntää samaa piiloluokkaa. [Kuva tähän?]

\subsection{Välimuistit}

(Inline cache)

\subsection{Roskienkeräys}

(Garbage collection)
\pagebreak
\section{Toteutusten vertailua}

Virtuaalikonetoteutukset ovat ottaneet käyttöön toistensa optimointimenetelmiä, mutta niiden kehittäjien erilaiset ajattelutavat ja perinteet ovat ohjanneet virtuaalikoneiden arkkitehtuureja ja käytäntöjä. Toteutuksista löytyy siis yhtäläisyyksiä sekä eroavaisuuksia, jotka vaikuttavat niiden suorituskykyyn, vaikka JavaScriptin toiminta on tarkasti määritelty. Vertailuun on valittu Googlen V8, Applen JavaScriptCore, Mozillan SpiderMonkey ja Microsoftin Chakra -virtuaalikoneet, koska niistä löytyy parhaiten tietoa paitsi Chakrasta. Se on toistaiseksi suljettua lähdekoodia, mutta silti mielenkiintoinen ja merkittävä käyttäjämäärällisesti. Microsoft on avaamassa sen lähdekoodin vuoden 2016 alussa~\cite{chakraopen}.

\subsection{Suoritusarkkitehtuuri}

Ensimmäinen kiinnostava tieto on, mitä virtuaalikone korkealla tasolla tekee koodille suorittaakseen sitä. Virtuaalikoneet eivät ole hylänneet tulkkejaan, vaikka niihin on lisätty JIT-kääntäjiä. Esimerkiksi JavaScriptCoren tulkki on kirjoitettu kokonaan uudelleen paljon edeltäjäänsä paremmaksi ja sille on annettu uusi nimi ''LLInt''~\cite{llint}. Ironista kyllä vaikka Google ei alunperin toteuttanut lainkaan tulkkia V8-virtuaalikoneeseensa, se on nyt kehittämässä tulkkia nimeltä ''Ignition''~\cite{v8ignition}. Uusi tulkki perustuu JavaScriptCoren LLInt-tulkkiin.

Perustelut muutokselle ovat ymmärrettäviä, sillä uutta tulkkia aiotaan käyttää ainakin alkuun ensisijaisesti mobiililaitteissa, joissa on rajoitettu määrä muistia ja heikompi suoritusteho. Tällaisilla laitteilla kääntäminen konekoodiksi ennen suorittamista yksinkertaisesti liian hidasta, minkä takia halutaan nopeasti käynnistyvä tulkki. Lisäksi käännetty koodi vie enemmän muistia kuin tavukoodimuotoinen koodi.

Aikaisemmin V8 on pitänyt JavaScript-lähdekoodia parhaana esitysmuotona ohjelmalle ja siitä on aina jäsennetty abstrakti syntaksipuu joka kerta ennen kääntämistä. Uusi tavukoodimuoto voisi kuitenkin toimia myös parempana välikielenä kääntäjille ja alkuperäistä lähdekoodia ei tarvitsisi jäsentää uudelleen joka kerta.

Myös Chakra virtuaalikone käyttää tulkkia suorituksen alussa. Uusimmassa Chakra-virtuaalikoneessa on optimoivan JIT-kääntäjän lisäksi yksinkertainen JIT-kääntäjä, joka ei tee monimutkaisia optimointeja mahdollistaen nopeamman käännöksen konekoodiksi~\cite{chakra}. Saatavilla olevien tietojen mukaan Chakra ei tue aktivaatiotietueiden manipulointia, joten optimoitu koodi otetaan käyttöön vasta kun funktiota kutsutaan uudestaan. Chakran kehittäjät ovat pyrkineet hyödyntämään laitteistoa mahdollisimman paljon rinnakkaistamalla käännös- ja roskienkeruuoperaatioita.

SpiderMonkeyn kehittäjät ovat olleet myös ahkeria ja virtuaalikone on nähnyt jo viisi erilaista JIT-kääntäjää: TraceMonkey, JägerMonkey, IonMonkey, OdinMonkey ja alkutilannekääntäjä (Baseline compiler)~\cite{monkeys}. Kaikki niistä ei ole enää käytössä, vaan osa on korvannut aikaisempia ja osa on siirtynyt eri vaiheeseen suorituksessa. Alkutilannekääntäjä on korvannut JägerMonkeyn, joka korvasi sitä edeltävän TraceMonkeyn.

SpiderMonkey käyttää yhä tulkkia suorituksen alussa, mutta pyrkii kääntämään koodin mahdollisimman nopeasti alkutilannekääntäjällään~\cite{baseline}, IonMonkey keskittyy erityisesti paljon kutsuttujen funktioiden kovakouraiseen optimointiin ja OdinMonkey-kääntäjää käytetään vain asm.js-muotoisen koodin kääntämiseen etukäteen. Asm.js:stä kerrotaan lisää Tulevaisuus-luvussa.

SpiderMonkey ei ole ainut virtuaalikone, joka on tuottanut useita virtuaalikoneita. V8:n kehittäjät rakentavat uutta JIT-kääntäjää, jota he kutsuvat nimellä TurboFan~\cite{turbofan}. JavaScriptCoren kehittäjät valmistelevat FTL JIT -nimistä kääntäjää~\cite{ftljit}.

\begin{comment}
\begin{itemize}
\item V8 - vain konekoodia - ei tulkkia, baseline, crankshaft(, turbofan) (TULKKI?!)
\item V8: \textbf{JS} -> AST -> native/Hydrogen (ssa) IR
%\item \url{https://docs.google.com/document/d/11T2CRex9hXxoJwbYqVQ32yIPMh0uouUZLdyrtmMoL44/edit#}
\item JSC - LLInt tulkki, method jit, dfg jit, ftl jit
\item JSC JS -> AST -> \textbf{tavukoodi} (IR) -> native
\item SpiderMonkey - tulkki, traceMonkey, jägerMonkey, ionMonkey
\item SpiderMonkey: JS -> AST -> tavukoodi -> native
%\item \url{http://www.infoq.com/news/2011/05/ionmonkey}
\item Chakra - tulkki, simple jit, full jit
\item Chakra: JS -> AST -> \textbf{tavukoodi}
%\item \url{https://channel9.msdn.com/Events/WebPlatformSummit/2015/Chakra-The-JavaScript-Engine-that-powers-Microsoft-Edge}
\item SSA-muoto ja de-facto optimoinnit!
\end{itemize}
\end{comment}

% https://wingolog.org/archives/2011/10/28/javascriptcore-the-webkit-js-implementation

%To put that into perspective, in V8, the high-level intermediate representation is the JS source code itself. When V8 first sees a piece of code, it pre-parses it to raise early syntax errors. Later when it needs to analyze the source code, either for the full-codegen compiler or for Hydrogen, it re-parses it to an AST, and then works on the AST.

%In contrast, in JSC, when code is first seen, it is fully parsed to an AST and then that AST is compiled to bytecode. After producing the bytecode, the source text isn't needed any more, and so it is forgotten. The interpreter interprets the bytecode directly. The simple method JIT compiles the bytecode directly. The DFG JIT has to re-parse the bytecode into an SSA-style IR before optimizing and producing native code, which is a bit more expensive but worth it for hot code.

\subsection{Tyyppien päättely}

Kaikki vertailun virtuaalikoneet käyttävät jonkinlaista tyyppijärjestelmää taustalla ja päättelevät tyyppejä suoritusaikana kerätyn tiedon perusteella. Konekoodista ei yksinkertaisesti saa nopeaa ilman tyyppispesifisiä käskyjä, koska ne ovat todella nopeita.

V8-virtuaalikoneen tapauksessa kerätty tyyppitieto tallennetaan sisällytettyihin välimuisteihin, jotka ovat osa generoitua konekoodia (eli suoritettavaa muistia)~\cite{llint}. JavaScriptCoren LLInt-tulkki kerää myös tyyppitietoa sisällytetyillä välimuisteilla, mutta se ei tallenna tyyppitietoa suoritettavan muistin sekaan.

Myös Chakra käyttää sisällytettyjä välimuisteja, mutta Chakran kehittäjät ovat vieneet optimoinnin askelta pidemmälle. Chakra samastaa tyyppejä niin sanotulla ''equivalent object type specilization''-menetelmällä~\cite{chakra}. Menetelmän avulla esimerkiksi aikaisemmin mainitut Piste- ja Ympyräoliot pystyvät käyttämään samaa sisällytettyä välimuistia, vaikka niillä on eri piiloluokat, koska niiden rakenne on tarpeeksi lähellä toisiaan.

\begin{comment}
\begin{itemize}
\item V8 inline cache KOODISSA
\item LLInt inlince cache MUISTISSA
%Note that in order to tier directly to the optimizing compiler, you need type information. Building the LLInt with the DFG optimizer enabled causes the interpreter to be instrumented to record value profiles. These profiles record the types of values seen by instructions that load and store values from memory. Unlike V8, which stores this information in executable code as part of the inline caches, in the LLInt these value profiles are in non-executable memory.
\item \url{https://wiki.mozilla.org/TypeInference}
\item SpiderMonkey: Staattisen ja dynaamisen päättelyn yhdistelmä!
\item \url{https://trac.webkit.org/wiki/JavaScriptCore#TypeInference}
\item V8: hidden classes
\item Chakra: kohta 12 min (Polymorphic inline caches (equivalent object type specilization))
%https://channel9.msdn.com/Events/WebPlatformSummit/2015/Chakra-The-JavaScript-Engine-that-powers-Microsoft-Edge}
\end{itemize}
\end{comment}
\section{Tulevaisuus}

Brendan Eichin mukaan JavaScriptistä on tullut jo kliseen omaisesti ''Webin konekieli''~\cite{webassembly}. JavaScript-ohjelmia suoritetaan käytännössä jokaisella alustalla ja kehittäjät ovat alkaneet tehdä kääntäjiä, jotka kääntävät muita ohjelmointikieliä, vanhoja tai uusia, JavaScriptiksi. Tämä lisää painetta parantaa virtuaalikoneiden suorituskykyä ja lisätä matalamman tason rajapintoja kääntäjäohjelmoijien hyödynnettäväksi.

Asm.js~\cite{asmjs} on epävirallinen standardi JavaScriptin osajoukosta, jota on mahdollista kääntää tehokkaaksi konekoodiksi. Sen idea on olla toimivaa JavaScript-koodia, mutta mahdollistaa tehokas kääntäminen suoraan konekoodiksi. Se onnistuu kertomalla virtuaalikoneelle, että koodi on asm.js-muotoista ja tarjoamalla tyyppivihjeitä. Tässä esimerkki kokonaislukujen summafunktiosta asm.js-koodina:
\begin{lstlisting}
"use asm"; // Kerrotaan, että koodi on asm.js-muotoista.
function kokonaislukujenSumma(x, y) {
  x = x|0; y = y|0;
  return (x + y)|0;
}
\end{lstlisting}
Esimerkissä kokonaislukuparametrista annetaan vihje tekemällä funktion alussa bittitason tai-operaatio: \texttt{x = x|0}. Jos \texttt{x}:n arvo on \texttt{undefined} tai jokin olio, operaatio muuttaa sen kielen standardin mukaisesti kokonaisluvuksi. Jos virtuaalikone tukee asm.js-kääntämistä, tyyppivihje kertoo virtuaalikoneelle, että parametri on aina kokonaisluku. Virheelliset tyypit voidaan huomata staattisilla työkaluilla suorittamatta koodia. JavaScriptin aritmetiikka toimii aina liukuluvuilla, joten sen takia myös summan tulos pitää muuttaa kokonaisluvuksi.

Mozillan OdinMonkey-kääntäjän lisäksi myös Googlen V8-virtuaalikoneen kehittäjät suunnittelevat tukea asm.js-optimoinneille TurboFan-kääntäjän avulla~\cite{turbofan}. Asm.js:n tueksi JavaScriptiin on tuotu lisää suorituskykyä parantavia toimintoja, kuten \textit{SIMD-käskyt}~\cite{webassembly}. SIMD tulee sanoista \textit{Single Instruction Multiple Data}, joka tarkoittaa suomeksi: ''Yksi käsky, useita data-alkioita''. SIMD-käskyjen avulla pystytään hyödyntämään prosessorien mahdollisuutta käsitellä useita data-alkioita, kuten vektoreita, yhdellä konekäskyllä.

Yleensä JavaScript-ohjelmat käyttävät vain yhtä säiettä, joten ne eivät hyödynnä moniytimisiä prosessoreja kovin hyvin. Vaikka kielessä on jo Web Worker -rajapinta, joka mahdollistaa ohjelman jakamisen rinnakkaisiin tehtäviin. Tehtävien välinen kommunikaatio tapahtuu viestinvälityksellä, joka on melko hidasta. Tukea rinnakkaisohjelmoinnille halutaan parantaa tuomalla \textit{SharedArrayBuffer}-rajapinta, eli jaettu taulukkopuskuri, ja atomiset operaatiot. Nämä yhdessä mahdollistavat matalan tason rinnakkaisohjelmoinnin, joka hyödyttää varsinkin raskasta laskentaa vaativia sovelluksia kuten pelejä.

Asm.js alkoi kokeellisena toteutuksena, mutta nyt selainvalmistajat ja standardoijat kehittävät yhdessä virallista Webin konekieltä, jota he kutsuvat nimellä WebAssembly~\cite{webassembly}. Sen on tarkoitus tarjota matalan tason binääriformaatti, jota kääntäjät voivat tuottaa. Koska WebAssembly on tiiviissä binäärimuodossa, sitä ei tarvitse purkaa, kuten pakattua JavaScript-koodia, eikä jäsentää uudelleen selaimessa. WebAssemblyn tavoite ei ole korvata JavaScript-koodia ja nykyistä kehitystapaa, vaan tarjota parempi tuki myös käännetyille ohjelmille, jotka aikaisemmin ovat toimineet helposti haavoittuvina selainlaajennuksina.

% TODO: SoundScript in V8, Transpilers!
\section{Yhteenveto}

\textbf{UUSIKSI}

Vaikka JavaScriptin suunnittelijoilla ei voinut olla käsitystä mihin kaikkeen JavaScriptia tultaisiin käyttämään, he onnistuivat luomaan hitin. Alkuvaiheessa kukaan ei varmaankaan osannut ennustaa kielen tulevaa menestystä, eikä menestys olisi ollut mahdollinen ilman toteuttajien innovaatioita.

Googlen innovatiivinen työ V8:n kanssa on kannustanut muita virtuaalikoneiden kehittäjiä parantamaan virtuaalikoneidensa suorituskykyä. Siirtyminen pelkästä tulkista eritasoisiin JIT-kääntäjiin on parantanut suorituskykyä huomattavasti aikaisempaan arkkitehtuuriin verrattuna.

Nykyinen trendi käyttää JIT-kääntäjiä ja olettaa ohjelmien staattinen käyttäytyminen voi kuitenkin olla huono idea pidemmällä tähtäimellä. Vaikka virtuaalikoneet näyttävät nopeilta suorituskykytesteissä, voi verkkopalveluiden todelliset käyttäytymismallit olla dynaamisempia. Korkean tason ohjelmointikielen ohjelmoijan ei pitäisi tarvita tietää virtuaalikoneen sisäisestä toteutuksesta pystyäkseen kirjoittamaan tehokasta koodia.

Virtuaalikoneiden kehittäjät tuntuvat jatkuvasti julkaisevan uusia viestejä blogeissaan, joissa he kertovat kuinka he ovat taas keksineet tai toteuttaneet uusia tapoja optimoida virtuaalikonettaan. Esimerkiksi V8:n kehittäjät kertovat blogissaan toteuttavansa uutta optimoivaa kääntäjää~\cite{turbofan}, joka pystyy optimoimaan enemmän erikoistapauksia kuin nykyinen ja mahdollistaa helpomman jatkokehityksen.

% V8 tiimi kertoo, että se on aikaisemmin keskittynyt vain raa'an suorituskyvyn parantamiseksi ja seuraava vaihe on keskittyä sovelluskehyksiin ja käytännön käyttötapauksiin!

% --- References ---

\bibliographystyle{babplain-lf}
%\bibliographystyle{babalpha-lf}
\pagebreak
\bibliography{references}


% --- Appendices ---

% uncomment the following

% \newpage
% \appendix
% 
% \section{Esimerkkiliite}

\end{document}