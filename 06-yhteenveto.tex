\section{Yhteenveto}

JavaScript on löytänyt tiensä monille eri alustoille ja se on kasvattanut suosiotaan kehittäjien keskuudessa. Kukaan tuskin osasi ennustaa JavaScriptin tulevaisuutta, kun se luotiin. Se on käytännössä korvannut Javan ja muut laajennuksiin perustuvat kielet selaimista. Tämä ei olisi mahdollista ilman virtuaalikonekehittäjien panosta suorituskyvyn parantamiseksi.

Googlen innovatiivinen työ V8:n kanssa on kannustanut muita virtuaalikoneiden kehittäjiä parantamaan virtuaalikoneidensa suorituskykyä. Siirtyminen pelkästä tulkista useisiin JIT-kääntäjiin on parantanut suorituskykyä huomattavasti aikaisempaan arkkitehtuuriin verrattuna.

Tämänhetkisten optimointimenetelmien riippuvuus staattisesta käytöksestä vähentää niiden hyödyllisyyttä todellisissa sovelluksissa. Ainakin Google kertoo siirtävänsä huomionsa raa'asta suorituskyvystä yleisiin käyttötapauksiin ja sovelluskehyksiin. Onkin tärkeää opetella käyttämään työkaluja, joilla oman sovelluksen suorituskykyä voi mitata, sen sijaan, että opettelisi ulkoa optimointikikkoja. Virtuaalikoneet muuttuvat niin nopeaa tahtia, että se mikä vielä eilen oli hidasta voi olla huomenna jo nopeaa.

Virtuaalikoneiden kehittäjät julkaisevat jatkuvasti blogiviestejä uusista ominaisuuksista ja optimointimenetelmistä. Microsoft kertoi juuri avaavansa oman toteutuksensa avoimeksi lähdekoodiksi ja näin auttaa kaikkia toteutuksia jakamalla ideoitaan. JavaScriptin tulevaisuus vaikuttaa kirkkaalta ja Brendan Eichin, JavaScriptin luojan, sanoihin on hyvä päättää: ''Always bet on JS''~\cite{beton}.

%Vaikka JavaScriptin suunnittelijoilla ei voinut olla käsitystä mihin kaikkeen JavaScriptia tultaisiin käyttämään, he onnistuivat luomaan \textbf{hitin}. Alkuvaiheessa kukaan ei varmaankaan osannut ennustaa kielen tulevaa menestystä, eikä menestys olisi ollut mahdollinen ilman toteuttajien innovaatioita.

%Googlen innovatiivinen työ V8:n kanssa on kannustanut muita virtuaalikoneiden kehittäjiä parantamaan virtuaalikoneidensa suorituskykyä. Siirtyminen pelkästä tulkista eritasoisiin JIT-kääntäjiin on parantanut suorituskykyä huomattavasti aikaisempaan arkkitehtuuriin verrattuna.

%Nykyinen trendi käyttää JIT-kääntäjiä ja olettaa ohjelmien staattinen käyttäytyminen voi kuitenkin olla huono idea pidemmällä tähtäimellä. Vaikka virtuaalikoneet näyttävät nopeilta suorituskykytesteissä, voi verkkopalveluiden todelliset käyttäytymismallit olla dynaamisempia. Korkean tason ohjelmointikielen ohjelmoijan ei pitäisi tarvita tietää virtuaalikoneen sisäisestä toteutuksesta pystyäkseen kirjoittamaan tehokasta koodia.

%Virtuaalikoneiden kehittäjät tuntuvat jatkuvasti julkaisevan uusia viestejä blogeissaan, joissa he kertovat kuinka he ovat taas keksineet tai toteuttaneet uusia tapoja optimoida virtuaalikonettaan. Esimerkiksi V8:n kehittäjät kertovat blogissaan toteuttavansa uutta optimoivaa kääntäjää~\cite{turbofan}, joka pystyy optimoimaan enemmän erikoistapauksia kuin nykyinen ja mahdollistaa helpomman jatkokehityksen.

% V8 tiimi kertoo, että se on aikaisemmin keskittynyt vain raa'an suorituskyvyn parantamiseksi ja seuraava vaihe on keskittyä sovelluskehyksiin ja käytännön käyttötapauksiin!