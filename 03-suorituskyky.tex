\section{Suorituskyky}

On selvää, että JavaScript-koodin kääntäminen konekoodiksi on aina nopeampaa kuin tulkkaaminen, sillä tulkkaamiseen kuluu aina enemmän konekäskyjä kuin valmiin koodin suorittamiseen. Toki kääntäminen tuo ongelmaksi hitaan käännösvaiheen. Tämän takia modernit virtuaalikoneet sisältävät useamman kuin yhden kääntäjän eritasoisilla optimoinneilla. Tässä luvussa esitellään muutamia keinoja, joilla JavaScript-virtuaalikoneet ovat parantaneet suorituskykyä.

\subsection{Piiloluokat}

\textit{Piiloluokat} (hidden classes)~\cite{v8design} ovat staattisia luokkia, joita virtuaalikone käyttää JavaScript-objektien kuvaamiseen. Jos objektia muutetaan, esimerkiksi lisäämällä uusi kenttä, luodaan sille uusi piiloluokka. Piiloluokkien hyödyt tulevat esiin, jos ohjelmassa on paljon samanlaisia objekteja, jotka voivat hyödyntää samaa piiloluokkaa. [Kuva tähän?]

\subsection{Välimuistit}

(Inline cache)

\subsection{Roskienkeräys}

(Garbage collection)