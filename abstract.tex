JavaScript-virtuaalikoneet ovat perinteisesti toimineet tulkkaamalla lähdekoodista muodostettua abstraktia syntaksipuuta tai tavukoodia. Kaikissa tutkituissa virtuaalikoneissa on otettu käyttöön JIT-kääntäjiä suorituskyvyn parantamiseksi, mutta tulkkia käytetään yhä. Varsinkin suorituksen alussa tulkki on JIT-kääntämistä parempi vaihtoehto vähäisen muistinkäytön ja nopean käynnistymisen ansiosta.

JavaScriptin dynaamisuudesta johtuen virtuaalikoneiden kehittäjät ovat joutuneet toteuttamaan monimutkaisia menetelmiä tehokkaan konekoodin tuottamiseksi JavaScript-ohjelmista. Monet optimointimenetelmät käyttävät hyväksi oletusta, että sovellukset eivät hyödynnä JavaScriptin dynaamisuutta liikaa, vaan käyttäytyvät melko staattisesti. Tämän oletuksen nojalla on pystytty hyödyntämään optimointimenetelmiä, joita tyypillisesti käytetään staattisesti tyypitettyjen kielten kanssa.

Oletus staattisesta käytöksestä on havaittu ongelmalliseksi. Vaikka yleisesti käytössä olevat suorituskykytestit käyttäytyvät varsin staattisesti, todelliset sovellukset hyödyntävät kielen dynaamisuutta enemmän, mikä heikentää optimointimenetelmien toimivuutta. Avoimuus ja kilpailu on auttanut virtuaalikoneiden suorituskyvyn kehitystä, mutta parannettavaa on vielä.